\documentclass[main.tex]{subfiles}

\begin{document}

\newrefcontext[sorting=hot]
%\printbibliography[filter={terrestrialplanets},keyword={primary},heading=bibintoc,title={\textcolor{beaver}{Biblio about: ''Sistema solare: pianeti terrestri''}}]
%\printbibliography[filter={terrestrialplanets},notkeyword={primary},heading=bibintoc,title={\textcolor{beaver}{Other refs about: ''Sistema solare: pianeti terrestri''}}]

{\let\clearpage\relax
\chapter{Helioseismology}
}

\begin{refsection}[solarsystem.bib]
\begingroup
\nocite{*}
\let\clearpage\relax
%\settoggle{bbx:note}{false}
%\AtNextBibliography{\AtEveryBibitem{\clearlist{note}}}
\boolfalse{isreasearch}
\printbibliography[filter={pps},heading=bibintoc,title={\textcolor{antiquefuchsia}{Vincoli a formazione planetaria da modelli/osservazioni formazione sistema solare}}]
\booltrue{isreasearch}
\printbibliography[filter={pps},check=bibsearchdef,heading=bibintoc,title={\textcolor{antiquefuchsia}{Bib research}}]
\endgroup
\end{refsection}

\section{CAI, Chondrules etc}

\section{Jupiter obseravtions and interior models}

\begin{enumerate}[label=(\alph*)]
    \item \cite{serra2016gravimetry}. High accuracy: relativistic perturbation. Dynamics involved in simulation - Spacecraft around Jupiter, baricenter of Jupiter system around the Sun. Gravity field of Jupiter expansion in spherical harmonics ($\Sigma_{BF}=O\hat{e}_1\hat{e}_2\hat{e}_3$, $\hat{e}_3\parallel\vec{\Omega}_J$,  body fixed RF): kaula 66. Gravitational potential at SC (spacecraft):
    \begin{equation*}
    U(r,\theta,\lambda)=\frac{GM_J}{r}\sum_l\sum_m^l\frac{R_J^l}{r^l}[C_{lm}\cos{(m\lambda)}+S_{lm}\sin{(m\lambda)}]P_{lm}(\sin{(\theta)})
    \end{equation*}
    Zonal coeff.: $C_{lm}, m=0$
    Tesseral coeff.: $C_{lm}, S_{lm}, 0<m<l$
    Greavitational moments of deg l: $J_l=-C_{l0}$ - opposite of zonal coeff. of deg l.
    Supponiamo asse di rotazione \'e la direzione che massimizza momento inerzia $I_{33}$;
    \begin{align*}
    &[\hat{e}_3]_{\Sigma_{EQ}}=(\sin{\delta_1}\cos{\delta_2},-\sin{\delta_2},\cos{\delta_1}\cos{\delta_2})\\
    &R(t)=R_3(\phi_t)R_1(\delta_2)R_2(\delta_1): \Sigma_{EQ}\to\Sigma_{BF}
    \end{align*}
    Perturbations: tidal deformation of Jupiter due to Galileian satellites (static tidal theory)
    \begin{align*}
    &U_{Love}=\sum_{l=2}^{\infty}\frac{k_lGM_PR^{2l+1}}{r_P^{l+1}r^{l+1}}P_l(\cos{\phi})
    \end{align*}
    $k_l$ measure jupiter response to perturbations, $\phi$ is the angle between perturbing body and orbiter.
    Lense-Thirring effect: central mass rotates endowed with angular momentum $\vec{J}$; precession of longitude of ascending node of body in gravity field of a central mass:
    \begin{align*}
    &\vec{a}_{LT}=\frac{(1+\gamma)G}{c^2r^3}[-\vec{J}\wedge\dvec{r}+\frac{3(\scap{J}{r})(\vec{r}\wedge\dvec{r})}{r^2}]
    \end{align*}
    \item \cite{helled2014measuring}.
    \item \cite{debras2019new}
    \begin{enumerate}[label=(\roman*)]
        \item Low value of high order gravitational moments and new for H/He denser in \si{\mega\bar} region (conflict with Galileo).
        \item Jupiter models satisfying Galileo/Juno contrains: 4 different regions (oter convective envelope, region of compositional changes, inner convective envelope, )
    \end{enumerate}
\end{enumerate}

{\let\clearpage\relax
\chapter{Struttura orbitale sistemi planetarii}
}

\begin{refsection}[orbitalevolution.bib,helio.bib]
\begingroup
\nocite{*}
\let\clearpage\relax
\printbibliography[filter={pps},heading=bibintoc,title={\textcolor{antiquefuchsia}{Evoluzione orbitale in PPD e dopo che \'e dissipato}}]
\endgroup
\end{refsection}



\end{document}